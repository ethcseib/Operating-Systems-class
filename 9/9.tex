\documentclass[12pt]{article}

%Packages add more power to LaTeX documents
\usepackage{fullpage} %Otherwise there will be a lot of wasted space at the margins
\usepackage{enumerate} %For the multi-part problem in example #4
\usepackage{amsthm} %For proof environment
\usepackage{amsmath} %For math symbols (like the black square)
\usepackage{graphicx,float,wrapfig} %Including graphics like PDFs and some image formats.
\usepackage{color,soul}
%\usepackage{qtree}
\author{Ethan Seiber}
\title{CSCI 485 assignment 9}


\begin{document}
\maketitle

\begin{enumerate}
\item[1] \hl{lecture 7 file systems p3 13:30}
	\begin{align*}%the number of blocks that exist in memory.
	\frac{2*2^{40}}{2*2^{10}}= 2^{30}\text{ blocks}
	\end{align*}
	\begin{align*}%The number of entries in one block. Which represents the number of free blocks in that block.
	\frac{2*2^{10}}{2^2} = 2^9\text{ entries in one block}
	\end{align*}
	
	\begin{align*}%number of blocks needed 
	\frac{2^{30}}{2^9} = 2^{21}\text{ blocks in total}
	\end{align*}
	
	\begin{align*}
	\frac{2^{21}*2^{11}}{2^{41}}*100 = \frac{1}{2^{19}}*100 = 0.1953125\%
	\end{align*}

      \item[2]
        \begin{align*}%Get the total number of blocks that can be made of the memory.
          \frac{2^{41}}{2^2*2^{10}} = 2^{29}
          \end{align*}

        \begin{align*}%Get the total number of blocks needed for free blocks.
          \frac{2^{29}}{2^3*2^{10}*2^2} = 2^{14}
        \end{align*}

        \begin{align*}%Get the percentage of memory used.
          \frac{2^{14}*2^2*2^{10}}{2^{41}}*100 = ~0.0031\%
          \end{align*}
\end{enumerate}
\end{document}
